\documentclass[10pt, letterpaper]{article}

% Packages:
\usepackage[
    ignoreheadfoot, % set margins without considering header and footer
    top=2 cm, % seperation between body and page edge from the top
    bottom=2 cm, % seperation between body and page edge from the bottom
    left=2 cm, % seperation between body and page edge from the left
    right=2 cm, % seperation between body and page edge from the right
    footskip=1.0 cm, % seperation between body and footer
    % showframe % for debugging 
]{geometry} % for adjusting page geometry
\usepackage[explicit]{titlesec} % for customizing section titles
\usepackage{tabularx} % for making tables with fixed width columns
\usepackage{array} % tabularx requires this
\usepackage[dvipsnames]{xcolor} % for coloring text
\definecolor{primaryColor}{RGB}{0, 79, 144} % define primary color
\usepackage{enumitem} % for customizing lists
\usepackage{fontawesome5} % for using icons
\usepackage{amsmath} % for math
\usepackage[
    pdftitle={Giovanni Santini's CV},
    pdfauthor={Giovanni Santini},
    pdfcreator={LaTeX with RenderCV},
    colorlinks=true,
    urlcolor=primaryColor
]{hyperref} % for links, metadata and bookmarks
\usepackage[pscoord]{eso-pic} % for floating text on the page
\usepackage{calc} % for calculating lengths
\usepackage{bookmark} % for bookmarks
\usepackage{lastpage} % for getting the total number of pages
\usepackage{changepage} % for one column entries (adjustwidth environment)
\usepackage{paracol} % for two and three column entries
\usepackage{ifthen} % for conditional statements
\usepackage{needspace} % for avoiding page brake right after the section title
\usepackage{iftex} % check if engine is pdflatex, xetex or luatex

% Ensure that generate pdf is machine readable/ATS parsable:
\ifPDFTeX
    \input{glyphtounicode}
    \pdfgentounicode=1
    \usepackage[T1]{fontenc}
    \usepackage[utf8]{inputenc}
    \usepackage{lmodern}
\fi

\usepackage[default, type1]{sourcesanspro} 

% Some settings:
\AtBeginEnvironment{adjustwidth}{\partopsep0pt} % remove space before adjustwidth environment
\pagestyle{empty} % no header or footer
\setcounter{secnumdepth}{0} % no section numbering
\setlength{\parindent}{0pt} % no indentation
\setlength{\topskip}{0pt} % no top skip
\setlength{\columnsep}{0.15cm} % set column seperation
\usepackage{etoolbox}
\makeatletter
\let\ps@customFooterStyle\ps@plain % Copy the plain style to customFooterStyle
\patchcmd{\ps@customFooterStyle}{\thepage}{
    \color{gray}\textit{\small Giovanni Santini - Page \thepage{} of \pageref*{LastPage}}
}{}{} % replace number by desired string
\makeatother
\pagestyle{customFooterStyle}

\titleformat{\section}{
    % avoid page braking right after the section title
    \needspace{4\baselineskip}
    % make the font size of the section title large and color it with the primary color
    \Large\color{primaryColor}
}{
}{
}{
    % print bold title, give 0.15 cm space and draw a line of 0.8 pt thickness
    % from the end of the title to the end of the body
    \textbf{#1}\hspace{0.15cm}\titlerule[0.8pt]\hspace{-0.1cm}
}[] % section title formatting

\titlespacing{\section}{
    % left space:
    -1pt
}{
    % top space:
    0.3 cm
}{
    % bottom space:
    0.2 cm
} % section title spacing

% \renewcommand\labelitemi{$\vcenter{\hbox{\small$\bullet$}}$} % custom bullet points
\newenvironment{highlights}{
    \begin{itemize}[
        topsep=0.10 cm,
        parsep=0.10 cm,
        partopsep=0pt,
        itemsep=0pt,
        leftmargin=0.4 cm + 10pt
    ]
}{
    \end{itemize}
} % new environment for highlights

\newenvironment{highlightsforbulletentries}{
    \begin{itemize}[
        topsep=0.10 cm,
        parsep=0.10 cm,
        partopsep=0pt,
        itemsep=0pt,
        leftmargin=10pt
    ]
}{
    \end{itemize}
} % new environment for highlights for bullet entries


\newenvironment{onecolentry}{
    \begin{adjustwidth}{
        0.2 cm + 0.00001 cm
    }{
        0.2 cm + 0.00001 cm
    }
}{
    \end{adjustwidth}
} % new environment for one column entries

\newenvironment{twocolentry}[2][]{
    \onecolentry
    \def\secondColumn{#2}
    \setcolumnwidth{\fill, 4.5 cm}
    \begin{paracol}{2}
}{
    \switchcolumn \raggedleft \secondColumn
    \end{paracol}
    \endonecolentry
} % new environment for two column entries

\newenvironment{threecolentry}[3][]{
    \onecolentry
    \def\thirdColumn{#3}
    \setcolumnwidth{1 cm, \fill, 4.5 cm}
    \begin{paracol}{3}
    {\raggedright #2} \switchcolumn
}{
    \switchcolumn \raggedleft \thirdColumn
    \end{paracol}
    \endonecolentry
} % new environment for three column entries

\newenvironment{header}{
    \setlength{\topsep}{0pt}\par\kern\topsep\centering\color{primaryColor}\linespread{1.5}
}{
    \par\kern\topsep
} % new environment for the header

\newcommand{\placelastupdatedtext}{% \placetextbox{<horizontal pos>}{<vertical pos>}{<stuff>}
  \AddToShipoutPictureFG*{% Add <stuff> to current page foreground
    \put(
        \LenToUnit{\paperwidth-2 cm-0.2 cm+0.05cm},
        \LenToUnit{\paperheight-1.0 cm}
    ){\vtop{{\null}\makebox[0pt][c]{
        \small\color{gray}\textit{Last updated in March 2025}\hspace{\widthof{Last updated in March 2025}}
    }}}%
  }%
}%

% save the original href command in a new command:
\let\hrefWithoutArrow\href

% new command for external links:
\renewcommand{\href}[2]{\hrefWithoutArrow{#1}{\ifthenelse{\equal{#2}{}}{ }{#2 }\raisebox{.15ex}{\footnotesize \faExternalLink*}}}


\begin{document}
    \newcommand{\AND}{\unskip
        \cleaders\copy\ANDbox\hskip\wd\ANDbox
        \ignorespaces
    }
    \newsavebox\ANDbox
    \sbox\ANDbox{}

    \placelastupdatedtext
    \begin{header}
        \fontsize{30 pt}{30 pt}
        \textbf{Giovanni Santini}

        \vspace{0.3 cm}

        \normalsize
        \mbox{{\footnotesize\faMapMarker*}\hspace*{0.13cm}Trento, IT}%
        \kern 0.25 cm%
        \AND%
        \kern 0.25 cm%
        \mbox{\hrefWithoutArrow{mailto:santigio2003@gmail.com}{{\footnotesize\faEnvelope[regular]}\hspace*{0.13cm}santigio2003@gmail.com}}%
        \kern 0.25 cm%
        \AND%
        \kern 0.25 cm%
        \mbox{\hrefWithoutArrow{tel:+39 3394072388}{{\footnotesize\faPhone*}\hspace*{0.13cm}+39 3394072388}}%
        \kern 0.25 cm%
        %\AND%
        %\kern 0.25 cm%
        %\mbox{\hrefWithoutArrow{https://yourwebsite.com/}{{\footnotesize\faLink}\hspace*{0.13cm}yourwebsite.com}}%
        %\kern 0.25 cm%
        %\AND%
        %\kern 0.25 cm%
        %\mbox{\hrefWithoutArrow{https://linkedin.com/in/yourusername}{{\footnotesize\faLinkedinIn}\hspace*{0.13cm}yourusername}}%
        %\kern 0.25 cm%
        \AND%
        \kern 0.25 cm%
        \mbox{\hrefWithoutArrow{https://github.com/San7o}{{\footnotesize\faGithub}\hspace*{0.13cm}San7o}}%
    \end{header}

    \vspace{0.3 cm - 0.3 cm}


\iffalse
    \section{Welcome to RenderCV!}



        
        \begin{onecolentry}
            \href{https://rendercv.com}{RenderCV} is a LaTeX-based CV/resume version-control and maintenance app. It allows you to create a high-quality CV or resume as a PDF file from a YAML file, with \textbf{Markdown syntax support} and \textbf{complete control over the LaTeX code}.
        \end{onecolentry}

        \vspace{0.2 cm}

        \begin{onecolentry}
            The boilerplate content was inspired by \href{https://github.com/dnl-blkv/mcdowell-cv}{Gayle McDowell}.
        \end{onecolentry}


    
    \section{Quick Guide}

    \begin{onecolentry}
        \begin{highlightsforbulletentries}


        \item Each section title is arbitrary and each section contains a list of entries.

        \item There are 7 unique entry types: \textit{BulletEntry}, \textit{TextEntry}, \textit{EducationEntry}, \textit{ExperienceEntry}, \textit{NormalEntry}, \textit{PublicationEntry}, and \textit{OneLineEntry}.

        \item Select a section title, pick an entry type, and start writing your section!

        \item \href{https://docs.rendercv.com/user_guide/}{Here}, you can find a comprehensive user guide for RenderCV.


        \end{highlightsforbulletentries}
    \end{onecolentry}

\fi
 
    \section{Education}



        
        \begin{threecolentry}{\textbf{BS}}{
            Sept 2022 – Now
        }
            \textbf{University of Trento}, Computer Science
            \begin{highlights}
                %\item GPA: 3.9/4.0 (\href{https://example.com}{a link to somewhere})
                \item \textbf{Coursework:} Computer Architecture, Operating Systems, Parallel Computing, Algorithms and Data Structures, Databases, Networking, Software Engineering, Calculus 1, Introduction to Computer and Network Security, Formal Languages and Compilers, Probability, Programming 1 and 2, Advanced Programming, Physics 1, Functioncal Programming, Linear algebra, Mathematical Foundations for Computer Science, Introduction to Web Programming, Logic, Introduction to Machine Learning
            \end{highlights}
        \end{threecolentry}

                
        \begin{threecolentry}{\textbf{}}{
            March 2021 – July 2021
          }
        \textbf{Cyberchallenge}, Cybersecurity CTF Course and Competition
            \begin{highlights}
                %\item GPA: 3.9/4.0 (\href{https://example.com}{a link to somewhere})
                \item I actively participated in the cybersecurity course \textit{Cyberchallenge} in 2021, held at the Università Politecnica delle Marche, achieving \textit{first place} in the internal competition and qualifying for the national Attack and Defense competition. The course introduced me to the fundamental concepts of cybersecurity with a hands-on approach through the CTF (Capture The Flag) format.
            \end{highlights}
        \end{threecolentry}

    
    \section{Projects}



        
        \begin{twocolentry}{
            \href{https://github.com/San7o/hive-operator}{github.com/San7o/hive-operator}
        }
            \textbf{Hive-ebpf kubernetes operator}
            \begin{highlights}
                \item I am developing an eBPF-based inode access logging tool for kubernetes clusters. The project is supervides by Bruno Crispo from the University of Trento.
                \item \textbf{Languages and Frameworks}: C, Go, Kubernetes, eBPF
            \end{highlights}
        \end{twocolentry}


        \vspace{0.1 cm}

        \begin{twocolentry}{
            \href{https://github.com/San7o/Baldo-Scanner}{github.com/San7o/Baldo-Scanner}
        }
            \textbf{Baldo Scanner}
            \begin{highlights}
                \item I developed an antivirus damenon written in C++ and a linux kernel module. It incorporates static malware analysis capabilities through signatures and rules, a simple kernel level firewall, a sandboxed execution environment, and a kernel module to collect information about calls to system calls. 
                \item \textbf{Languages and Framewroks}: C, C++, kprobes, netlink, character devices, sqlite, cmake
            \end{highlights}
        \end{twocolentry}

        \vspace{0.1 cm}
        
        \begin{twocolentry}{
            \href{https://github.com/San7o/santOS}{github.com/San7o/santOS}
        }
            \textbf{santOS}
            \begin{highlights}
                \item A general purpose operating system, currently supporting the i386 architecture. This project aims to build a full operating system with networking, filesystem, scheduler, IPC and userspace on multiple architectures. Currently the project is in Its early stages and I will keep working on It during my spare time.
                \item \textbf{Languages and Framewroks}: C, bios, make 
            \end{highlights}
        \end{twocolentry}


        

\iffalse
    \section{Experience}



        
        \begin{twocolentry}{
            Cupertino, CA

        June 2005 – Aug 2007
        }
            \textbf{Apple}, Software Engineer
            \begin{highlights}
                \item Reduced time to render user buddy lists by 75\% by implementing a prediction algorithm
                \item Integrated iChat with Spotlight Search by creating a tool to extract metadata from saved chat transcripts and provide metadata to a system-wide search database
                \item Redesigned chat file format and implemented backward compatibility for search
            \end{highlights}
        \end{twocolentry}


        \vspace{0.2 cm}

        \begin{twocolentry}{
            Redmond, WA

        June 2003 – Aug 2003
        }
            \textbf{Microsoft}, Software Engineer Intern
            \begin{highlights}
                \item Designed a UI for the VS open file switcher (Ctrl-Tab) and extended it to tool windows
                \item Created a service to provide gradient across VS and VS add-ins, optimizing its performance via caching
                \item Built an app to compute the similarity of all methods in a codebase, reducing the time from $\mathcal{O}(n^2)$ to $\mathcal{O}(n \log n)$
                \item Created a test case generation tool that creates random XML docs from XML Schema
                \item Automated the extraction and processing of large datasets from legacy systems using SQL and Perl scripts
            \end{highlights}
        \end{twocolentry}


\fi
    
    \section{Certifications}

        
        \begin{samepage}
            \begin{twocolentry}{
                Dec 2024
            }
                \mbox{NVIDIA}, \textbf{Fundamentals of Accelerated Computing with OpenACC}, Hackathon 1º place

                \vspace{0.10 cm}
                \begin{highlights}
                \item I partecipated in a two days workshop introducing the basics of accelerated computing using OpenACC, a powerful directive-based programming model. I learned how to optimize and parallelize code to fully leverage the capabilities of modern GPUs and CPUs.
                  \item The event ended with an hackathon about optimizing a machine learning model, I achieved first place.
                \end{highlights}
                \vspace{0.10 cm}

            \end{twocolentry}


              
        \end{samepage}


\iffalse
    \section{Technologies}



        
        \begin{onecolentry}
            \textbf{Languages:} C++, C, Java, Objective-C, C\#, SQL, JavaScript
        \end{onecolentry}

        \vspace{0.2 cm}

        \begin{onecolentry}
            \textbf{Technologies:} .NET, Microsoft SQL Server, XCode, Interface Builder
        \end{onecolentry}
\fi

    

\end{document}
